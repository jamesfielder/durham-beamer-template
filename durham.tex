\documentclass[xcolor=dvipsnames,usepdftitle=false]{beamer} 
%\usecolortheme[named=BlueViolet]{structure} 
\usecolortheme[RGB={83,13,88}]{structure} \usetheme[height=7mm]{Boadilla} 
\setbeamertemplate{items}[ball] 
\setbeamertemplate{navigation symbols}{}%remove navigation symbols 

\usefonttheme[onlylarge]{structuresmallcapsserif}
\usefonttheme{professionalfonts}

% Normal usepackage includes
\usepackage{graphicx}
\usepackage{color}
\usepackage{amsfonts}
\usepackage{amssymb}
\usepackage{textcomp}
\usepackage{amsthm}
\usepackage{amsopn}
\usepackage{subfigure}
\usepackage{fancybox}
\usepackage{mathtools}
%Set up command for logo in top right corner
\usepackage[absolute,overlay]{textpos}
\setlength{\TPHorizModule}{1mm}
\setlength{\TPVertModule}{1mm}
\newcommand{\MyLogo}{%
	\begin{textblock}{14}(119.0,1.0) %logo position
		\includegraphics[width=0.8cm]{pictures/logo_only.eps}
	\end{textblock}
}

% This puts in a section ``title slide'' when \section command
% is encountered.  Can easily be changed to subsection
\AtBeginSection{
\begin{frame}
		\MyLogo
\begin{center}
		\structure{\Large \insertsection}
	\end{center}
\end{frame}
}

%Use some script greek letters
\renewcommand\epsilon{\varepsilon}
\renewcommand\phi{\varphi}
\renewcommand\theta{\vartheta}

\begin{document}

\title{Durham University Beamer Template}
\author{}
\institute{Durham University}
\date{}

\begin{frame}[plane]

\titlepage

\end{frame}

\section{Maths Crap}

\begin{frame}

\frametitle{Some Maths Here}

The Crank-Nicolson scheme is
\begin{equation} \label{crank}
	U^{n+1}_{j} - U^{n}_{j} = \frac{\mu}{2} \left(U^{n+1}_{j+1} - 2U^{n+1}_j + U^{n+1}_{j-1} + U^n_{j+1} - 2U^n_j + U^{n}_{j-1}\right)
\end{equation}
or as a matrix equation
\begin{equation}
	\mathbf{A} U^{n+1} = \mathbf{B} U^n
\end{equation}
\end{frame}
\begin{frame}
Where $\mathbf{A}$ and $\mathbf{B}$ are
\[
A =\begin{pmatrix}
1-\mu & - \frac{\mu}{2} & 0 & \cdots & 0 \\
- \frac{\mu}{2} & 1-\mu & - \frac{\mu}{2} & \ddots & \vdots \\
0 & -\frac{\mu}{2} & 1-\mu & \ddots & 0 \\
\vdots & \ddots & \ddots & \ddots & -\frac{\mu}{2} \\
0 & \cdots & 0 & - \frac{\mu}{2} & 1-\mu
\end{pmatrix}
\]
and
\[
B =\begin{pmatrix}
1+\mu & \frac{\mu}{2} & 0 & \cdots & 0 \\
\frac{\mu}{2} & 1+\mu & \frac{\mu}{2} & \ddots & \vdots \\
0 & \frac{\mu}{2} & 1+\mu & \ddots & 0 \\
\vdots & \ddots & \ddots & \ddots & \frac{\mu}{2} \\
0 & \cdots & 0 & \frac{\mu}{2} & 1+\mu
\end{pmatrix}
\]

\end{frame}

\end{document}